\capitulo{6}{Trabajos relacionados}

En este apartado tendremos la oportunidad de hablar acerca de trabajos similares al que hemos realizado en este proyecto. 

Ya que no hay ninguno que se relacione con trabajos de la universidad de Burgos, hablaremos sobre los encontrados en Internet y los que se realizan en la empresa HP.

Al comenzar el trabajo final, el tutor académico me pregunto sobre mi conocimiento de Blockchain y la programación con Solidity, al ser principiante con esta tecnología me insto a seguir estos cursos para empezar a familiarizarme con este tipo de programación:

\paragraph{\textit{CryptoZombies}}\cite{crypto}: es una página, en la cual, tienes que seguir una serie de capítulos en la cual te enseñan la programación para la creación de contratos inteligentes en Solidity. 

Enseñan los conceptos básicos y los tipos de variables que hay, a demás de como implementar funciones y poder ejecutarlas.


\paragraph{\textit{Pet-shop}}\cite{shop}: tenemos un tutorial, el cual, nos explicará como poder crear una Dapp, y sera un sistema para poder adoptar mascotas. Nos enseña la utilización de:

\begin{itemize}
	\item Configurar el desarrollo en el que trabajaremos.
	\item Configurar un proyecto con Truffle.
	\item Escribir nuestro propio \textit{smart contract}.
	\item Compilación del \textit{smart contract}.
	\item Migrar y ejecutar el \textit{smart contract}.
	\item Crear la interfaz de usuario.
	\item Interactuar con metamask para dar los pasos para la compra de la mascota.
\end{itemize}

\paragraph{\textit{Crypto Kitties}}: es una de una de las grandes aplicaciones con mas fama en el mundo de las \textit{DApps}.

Con esta aplicación tendremos una familia de gatos ``digitales''\cite{gatos}, y el gato que se forme podrá valer una cierta cantidad dependiendo sus cualidades. 

Cabe destacar que cada gato en esta página es un \textit{token} indivisible de Ethereum, así que cualquier compra, venta o transacción que se pueda realizar conllevara un  gasto de minería.

Para poder interacutar con esta aplicación tendremos que tener instalado el plugin de Metamask y una cuenta con Ether suficiente para poder realizar las operaciones deseadas.

\paragraph{\textit{Brave}}\cite{brave}: es un navegador web de código abierto el cual esta basado en \textit{Chromium}\footnote{\textit{Chromium}:\cite{chromium} es un navegador web gratuito y de código abierto desarrollado por Google.}.

Este navegador tiene la capacidad de bloquear los anuncios(incluso lleva una cuenta de cuantos anuncios bloqueo) y una gran cantidad de rastreadores en linea, y asegura proteger la privacidad de los usuarios, reduciendo los datos con los anunciantes.

Según anuncian en la sus redes \textit{Brave} es mas rápido que sus competidores, Chrome y Safari, en ordenadores es dos veces mas rápido y en dispositivos móviles puede llegar a ser hasta ocho veces mas rápido.

Tiene también versión para móvil en las versiones de Android y iOS.

\paragraph{\textit{Aragon}}: es un proyecto que pretende eliminar los intermediarios en el proceso de creación y mantenimiento de estructuras organizativas. Esto es posible a través del uso de la cadena de bloques.


\section{Empresas y organizaciones que usan \textit{blockchain}}

Hay múltiples empresas que han empezado a utilizar la tecnología \textit{blockchain} en el día a día para mejorar la gestión u ofrecer al publico un servicio de más calidad.

A continuación, expondremos algunos ejemplos de estas empresas\cite{empresas}, las cuales las podremos dividir en diferentes sectores como: 
\subsection{Comercio}
\begin{itemize}
	\item \underline{\textit{Walmart}}: es una multinacional que quiere aprovechar el registro digital para mejorar las gestiones y los seguimientos de datos para usar en el día a día. 
\end{itemize}
\subsection{Envío de paquetería}
\begin{itemize}
	\item \underline{\textit{FedEx}}: una empresa de envíos que permite mejorar el proceso de resolución de disputas con los clientes. Quiere usarlo para identificar el tipo de datos, almacenarlo en una \textit{blockchain} y también para almacenar registros en futuros envíos o problemas.
\end{itemize}

\subsection{Protección de datos}
\begin{itemize}
	\item \underline{\textit{Facebook}}: explora el uso de \textit{blockchain} para mejorar la seguridad de los datos y poder ofrecer mayor seguridad a los usuarios.
	\item \underline{\textit{Google}}: realiza la misma operación que Facebook para poder proporcionar seguridad a todos sus usuarios.
\end{itemize}
\subsection{Transporte}
\begin{itemize}
	\item \underline{\textit{British Airways}}: lo realiza para administrar la información sobre los vuelos mas importantes en el aeropuerto de Londres y Miami. Con ello verifican la identidad de las personas que viajan en el vuelo, mediante la base de datos que se encuentra en \textit{blockchain}. 
	\item \underline{\textit{Toyota}}: para mejorar la conducción autónoma. 
\end{itemize}
\subsection{Bancos}
\begin{itemize}
	\item \underline{\textit{MasterCard}}: la compañía de \textbf{EE. UU.} lo utiliza para realizar los pagos de forma instantánea.
    \item \underline{\textit{BBVA}}:  también quiere conseguir la rápida circulación de la moneda mediante la transferencia. 
	\item \underline{\textit{Banco Santander}}: tiene una aplicación que ejecuta pagos internacionales, mediante los dispositivos móviles (\textit{smartphones}). La transferencia puede ser efectiva en torno a los 40 segundos. 
\end{itemize}






