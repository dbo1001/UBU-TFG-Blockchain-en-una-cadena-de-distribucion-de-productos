\capitulo{1}{Introducción}

La empresa HP con sede en León tiene como objetivo el desarrollo de software para impresoras de largo formato. Los productos que se investigan y desarrollan en este centro están relacionados con el \textit{firmware} de las máquinas así como con los programas que permiten el uso de las mismas a aplicaciones y sistemas operativos.

En los últimos tiempos, HP ha creado una nueva unidad de trabajo a la cual se la denomina grupo de innovación y consiste en explorar nuevas tecnologías. Una de las tecnologías a investigar es la tecnología Blockchain, de la que explicaremos en este documento. 

Blockchain\cite{informacionB} surge en 1991 de la mano de \textit{Stuart Haber} y \textit{W. Scott Stornetta}, ambos científicos crearon la tecnología de las criptomonedas más conocida como cadena de bloques (esto permite a cada cliente de la red poder hacer una transferencia a otra persona sin tener que confiar entre sí). Para dar más seguridad, un año después se incorporan los árboles Merkle\footnote{Árbol de \textit{Merkle}: es una estructura de árbol, en el que cada nodo que no es una hoja le corresponde una etiqueta hash de la concatenación de las etiquetas de los nodos hijo.}, lo que hace esta tecnología es que sea más eficaz y rugoso ya que permite que varios archivos o documentos se puedan juntar todos en un mismo bloque. En el año 2004, se introducen los sistemas RPoW (Prueba de trabajo Reutlizable), que resuelve el problema de doble gasto, en el 2008 se crea la red Bitcoin y en 2013 nace la red Ethereum.

Uno de los objetivos del presente proyecto es la creación de una página web, esta página la realizaremos en un servidor local llamado XAMPP, primeramente tendremos que instalar XAMPP (se explicará en el apartado de los anexos) esto será nuestra base de datos, con la cual guardaremos a los usuarios que se registren desde la página, desde la página tendremos opción de modificar y borrar los usuarios. 

Cuando el usuario se haya registrado, tendremos la posibilidad de realizar los smart contract, es un contrato que es capaz de ejecutarse y hacerse cumplir por sí mismo, de manera autónoma y automática, sin intermediarios ni mediadores, que conectará a un sistema de red privada llamada Rospten, y le pasaremos diferentes parámetros, una vez creado no se podrán ni modificar ni destruir. 

La realización de todas las pruebas ha ido cambiando desde la creación de diferentes \textit{smart contract} tanto en la red local (Ganache), en cuentas privadas (Ropsten) e incluso en la red principal de Ethereum. La realización de los \textit{smart contract} es mediante metamask, web3, node.js, solidity y más aplicaciones que explicaremos más detalladamente en los siguientes puntos.

\subsection{Estructura de la memoria}

Esta memoria constará de los contenidos explicados a continuación:

\begin{enumerate}[1)]
\item \textbf{Resumen}: breve introducción de los objetivos y descripción de la empresa con la cual realizo dicho trabajo. Este apartado estará tanto en castellano como en inglés. 
\item \textbf{Objetivos del proyecto}: se detallarán los objetivos qeu se quieren conseguir al realizar este proyecto.
\item \textbf{Conceptos teóricos}: realizamos un pequeño resumen de los conceptos que hemos aprendido al realizar dicho proyecto.
\item \textbf{Técnicas y herramientas}: descripción de los programas utilizados para la elaboración y seguimiento del proyecto.
\item \textbf{Aspectos relevantes del desarrollo del proyecto}: aclaración de las herramientas utilizadas y argumentación final de nuestro proyecto.
\item \textbf{Trabajos relacionados}: consiste en hablar de los trabajos similares a nuestro trabajo, tanto realizados por la universidad de Burgos o ajenos a ella.
\item \textbf{Conclusiones y líneas de trabajo futuras}: valoración de todo lo conseguido y posibles mejoras que se podrán mejorar en futuros trabajos relacionados con este proyecto.
\item \textbf{Bibliografía}: información encontrada en Internet, a la que se hace referencia en cada apartado usado en la memoria del proyecto. 
\end{enumerate}

\subsection{Estructura de los anexos}

Los anexos se estructuran de la siguiente manera:

\begin{enumerate}[1)]
	\item \textbf{Plan de proyecto de software}: estudio de la viabilidad tanto legal como económica del proyecto, también comentaremos los objetivos propuestos en cada reunión.
	\item \textbf{Especificación de usuarios}: en este apartado explicaremos los
casos de uso y requisitos funcionales de la herramienta creada.
	\item \textbf{Especificación de diseño}: se detallaran los diseños software utilizados en este proyecto.
	\item \textbf{Documentación técnica de programación}: explicación de la instalación del programa y guia del código fuente mas relevante.
	\item \textbf{Manual de usuario}: guía para el manejo de la aplicación para cualquier usuario, requisitos e instalación.
\end{enumerate}

\subsection{Contenido entregable en el CD}

Los documentos que contiene el CD serán:

\begin{enumerate}
\item Documentación de la memoria: versión en \textit{pdf} de la memoria final.
\item Documentación de los anexos: versión en \textit{pdf} de la anexos finales.
\item Código del programa: version final del código ejecutable para la puesta en marcha del trabajo.
\item Vídeo explicativo de las funcionalidades de la página web.
\end{enumerate}

